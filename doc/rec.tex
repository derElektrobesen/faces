\documentclass[a4paper]{article}

\usepackage[12pt]{extsizes}

\usepackage{mathtext}

\usepackage[T2A]{fontenc}
\usepackage[utf8]{inputenc}

\usepackage[english,russian]{babel}

\usepackage{indentfirst}
\usepackage{scrextend}
\usepackage{fancyhdr}
\newcommand*\rot{\rotatebox{90}}

\usepackage{xr}
\usepackage{url}
\usepackage{verbatimbox}
\usepackage{caption}
\usepackage{graphicx}

\usepackage{gensymb}
\usepackage{tabu}

\usepackage{listings}
\usepackage{color}
\definecolor{mygreen}{rgb}{0,0.6,0}
\definecolor{mygray}{rgb}{0.5,0.5,0.5}
\lstset{extendedchars=\true,
		breaklines=true,
		breakatwhitespace=true,
		captionpos=b,
		keepspaces=true,
		keywordstyle=\color{blue},
		numbers=left,
		tabsize=4,
		numberstyle=\tiny\color{mygray},
		commentstyle=\color{mygreen},
		language=C,
		basicstyle=\ttfamily,
		showstringspaces=false,
		title=\lstname
		}
 
\usepackage{geometry} % Меняем поля страницы
\geometry{left=2.8cm}% левое поле
\geometry{right=2.8cm}% правое поле
\geometry{top=2.5cm}% верхнее поле
\geometry{bottom=3cm}% нижнее поле

\usepackage{perpage}
\usepackage{placeins} % for float barriers
\usepackage{float}
\MakePerPage{footnote}

\setcounter{tocdepth}{2}
\ifx
Выставление глубины оглавления
n=4 это chapter, section, subsection, subsubsection и paragraph;
n=3 это chapter, section, subsection и subsubsection;
n=2 это chapter, section, и subsection;
n=1 это chapter и section;
n=0 это chapter.
\fi

\renewcommand*\thesection{\arabic{section}.}
\renewcommand{\thesubsection}{\thesection\arabic{subsection}}
\renewcommand{\thetable}{\thesection\arabic{table}}
\renewcommand{\theequation}{\thesection\arabic{equation}}

\newcommand{\newpar}{\par\medskip}
\usepackage{multirow,tabularx}

\newcommand{\sbt}{
	\,\begin{picture}(-1,1)(-1,-3)\circle*{4}\end{picture}\ 
}
\newcommand{\sbti}{\sbt~~}

\newcommand{\linux}{Linux}
\newcommand{\linuxv}{\linux\ v.3.9.2\ }
\newcommand{\gnu}{GNU}
\newcommand{\gnulinux}{\gnu/\linux}
\newcommand{\archlinux}{Arch Linux}
\newcommand{\fulllinux}{\archlinux, ядро \linuxv}
\newcommand{\unix}{UNIX\ }

\newcommand{\linuxpath}[1]{\texttt{#1}}
\newcommand{\linuxutil}[1]{\texttt{#1}}
\newcommand{\linuxcommand}[1]{\texttt{#1}}
\newcommand{\src}[1]{\linuxcommand{#1}}
\newcommand{\key}[1]{\src{<#1>}}

\begin{document}
\setlength{\parskip}{0.3cm}
\thispagestyle{empty}

\begin{center}
    \textbf{Рецензия} \\
    на квалификационную работу бакалавра по специальности \\
    <<Программное обеспечение ЭВМ и информационные технологии>> \\
    студента МГТУ~им.~Н.\,Э.\,Баумана \\
    Бережного Павла Юрьевича \\
    на тему: <<Распознавание человеческого лица по фотоснимку на мобильной \\
    платформе под управлением ОС Android>>.
\end{center}
\vspace{0.5cm}

Рецензируемая дипломная работа посвящена разработке алгоритма обнаружения и распознавания лиц и его программной реализации.
Направление распознавания лиц является востребованным в связи с возросшей актуальностью повышения информационной
безопасности инфокоммуникационных систем.

Пояснительная записка к дипломной работе состоит из технического задания, календарного плана, введения, четырёх основных
частей, заключения и списка литературы. В аналитическом разделе произведена постановка задачи, выполнен анализ предметной области,
анализ известных методов обнаружения и распознавания и обоснован выбор методов, используемых в комбинированном подходе.

В конструкторском разделе приведены описания выбранных методов обнаружения и распознавания лиц. Раздел оформлен правильно
и богат на иллюстрации. Алгоритмы описываются и словами, и с помощью блок-схем. Благодаря этому все алгоритмы просты для
понимания.

В технологическом разделе приведены обоснование выбранных средств и технологий разработки, диаграммы классов, требования к архитектуре и описание процесса
тестирования разработанного программного комплекса.

В исследовательском разделе описаны эксперименты, исследующие скорость, точность
и полноту обнаружения и распознавания лиц. Результаты экспериментов показывают то, что разработанное
решение работает и показаны преимущества разработанного решения над существующими.

Заключение дипломной работы кратко подводит итоги работы и описывает 
перспективы ее развития. В заключении приведены обоснованные выводы по работе. 

Большой размер списка использованной при написании дипломной работы 
литературы говорит о высоком качестве работы. Список использованной литературы 
состоит как из русскоязычных, так и из иностранных источников, что показывает 
высокую степень проработанности и изучения темы работы. 

Дипломная работа грамотно оформлена, соответствует стандартам ГОСТ. Она 
содержит большое количество табличного и иллюстративного материала, что 
позволяет более наглядно раскрыть ее основные результаты. 

Наряду с указанными достоинствами, дипломная работа не лишена недостатков. 
В качестве основного недостатка можно отметить то, что разработанный алгоритм распознавания
лиц не позволяет работать с фотоснимками, на которых запечатлены несколько человек.
Однако данный недостаток описывается автором и не носит существенного характера.

Представленная работа выполнена полностью в 
соответствии с предъявляемыми требованиями. Разработанный алгоритм успешно
находит и идентифицирует человека, изображенного на фотоснимке. Автор демонстрирует умение 
глубоко исследовать тему и конструировать собственные алгоритмы для решения 
возникающих задач.

В целом можно заключить, что дипломная работа Бережного Павла Юрьевича
рекомендована к защите и заслуживает оценки <<отлично>>, а её автор, Бережной Павел Юрьевич,
присвоения степени бакалавра техники и технологий.
\thispagestyle{empty}

\vspace{10cm}

\begin{flushleft}
    Рецензент:\\
    кандидат технических наук, \\
    технический директор Почты@Mail.Ru \\
    \begin{tabular}{cc}
        \underline{\hspace{4cm}} & Аникин~Д.\,Ю. \\
        (подпись) & (Ф.\,И.\,О.)
    \end{tabular}

    \vspace{2cm}
    \begin{tabular}{cc}
        Подпись Аникина~Д.\,Ю. удостоверяю & \underline{\hspace{7cm}} \\
                                           & Начальник отдела кадров \\
                                           & ООО <<Мэйл.Ру>> Сидорова~О.\,В.
\end{tabular}
\end{flushleft}
\end{document}
