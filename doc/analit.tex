\section{Аналитический раздел}

Проблема обнаружения и распознавания человеческого лица по фотоснимку стояла
перед разработчиками уже очень давно и ее решали многими способами.
В данном разделе будут рассмотрены основные методы обнаружения и распознавания
лиц, а так же их достоинства и недостатки.

\subsection{Существующие методы и программные продукты}

Проблема распознавания лиц рассматривалась еще на ранних
стадиях компьютерного зрения. Поэтому
существует множество методов для обнаружения и распознавания человеческого
лица по фотоснимку.
Ряд компаний на протяжении более 40 лет активно разрабатывают
автоматизированные, а сейчас и автоматические системы распознавания
человеческих лиц, например 
ImageWare (система FaceID \cite{faceid}) и Imagis Technologies Inc. (система
ID-2000 Image Detection and Biometric Facial Recognition \cite{imagis}).

Основными характеристиками, которыми различаются указанные системы, являются:
\begin{itemize}
    \item возможность поиска и распознавания нескольких лиц;
    \item устойчивость к изменениям в прическе, наличию/отсутствию усов
        и бороды, очкам (кроме солнцезащитных), возрастным
        изменениям (кроме детей), поворотам (до и более 30 градусов)
    \item возможность многокадрового анализа видеопотока,
        обеспечивающего повышение точности распознавания;
    \item слабая зависимость скорости работы от размера используемой
        галереи лиц. Например, при увеличении галереи со 100 до
        1000 лиц, скорость работы уменьшается менее чем на 10\%;
    \item работа с видео в режиме реального времени.
\end{itemize}

Указанные системы являются закрытыми для рядового пользователя и разработчика.
Однако существует и набор программных продуктов, распространяющихся по
открытой лицензии BSD, среди которых находится система OpenCV~\cite{opencv}.

OpenCV~-- библиотека компьютерного зрения с открытым исходным кодом
(Open Source Computer Vision Library), содержащая более 500
функций, заточенных под выполнение в реальном времени.

Библиотека содержит алгоритмы для обработки, реконструкции и
очистки изображений, распознания образов, захвата видео,
слежения за объектами, калибровки камер и др.

Открытая лицензия для OpenCV была составлена таким образом, чтобы было
возможно создавать коммерческие приложения, используя
любые возможности OpenCV. 
Отчасти из-за таких условий существует
большое сообщество пользователей, включающее в себя такие крупные
компании как IBM, Microsoft, Intel, Sony,
Siemens, Google, и это далеко не полный список, а также научно-исследовательские
центры, такие как Стэнфорд, Массачусетский технологический
институт, CMU, Кембридж, и INRIA. OpenCV популярна во всём мире,
причём большие сообщества пользователей можно найти в Китае,
Японии, России, Европе и Израиле.

Однако, все рассмотренные программные продукты являются громоздкими, реализующими
целый набор функций, многие из которых не являются актуальными
для данной работы.

В связи с этим, необходимо разработать новый программный продукт, который
является более легковесным и более специализированным по сравнению с
рассмотренными продуктами. Он должен рабоать достаточно производительно под
управлением мобильной операционный системы Android.

Пользователи продукта не должны иметь никаких специализированных навыков,
которые нужны, например, при работе с библиотекой OpenCV. Интерфейс
программного продукта должен быть таким, чтобы пользователям было комфортно
и они не ощущали трудностей при его изучении. При этом продукт должен быть
доступен для скачивания рядовому пользователю.

\subsection{Обнаружение лиц}

Существует множество методов для обнаружения человеческого лица
на растровом изображении.
В таблице \ref{tables:finding_methods} представлены сущесвующие
группы алгоритмов обнаружения.

\begin{table}[hbt]
    \centering
    \begin{tabu}[\textwidth]{|X[c]|X[c]|X[c]|}
        \hline
        Название группы методов & Преимущества & Недостатки \\
        \hline
        Поиск по цвету & Высокая скорость обнаружения,
            возможность селекции по цвету кожи, не требует обучения &
            Очень высокая вероятность ложного срабатывания \\
        \hline
        Корреляционные методы & Высокая точность обнаружения &
            Крайне низка скорость работы, требуется большая обучающая выборка
            с жесткими условиями съемки \\
        \hline
        Поиск по характерным точкам лица & Высокая скорость и точность
            обнаружения для изображений лиц круглым планом &
            Низкая точность и скорость обнаружения для изображений с несколькими
            лицами, низкая точность для менее качественных изображений \\
        \hline
    \end{tabu}
    \caption{Сравнение существующих групп алгоритмов обнаружения
        человеческого лица}
    \label{tables:finding_methods}
\end{table}

Одним из алгоритмов, базирующихся на поиске лица по интенсивности цвета, является
метод Виолы-Джонса \cite{viola_jones_2}.

Метод реализован в таких проектах, как <<HxMarilena>>, <<OpenClooVision>>, <<MATLAB: Viola-Jones
Object Detection>>, <<OpenCV>>, и других.

\subsection{Определение необходимых эксплуатационных свойств разработки}

В этом разделе будут рассмотрены требования к конечному программному продукту,
цели и задачи, которые ставит перед собой разработчик, а так же входные
и выходные данные и параметры приложения.

\subsubsection{Выполняемые задачи}

Разработанное приложение должно корректно определять на цифровом фотоснимке
местонахождение человеческого лица, а так же по возможности определять
кто изображен на снимке. При этом не подразумевается, что пользователь
ожидает увидеть именно имя человека. Допускается использование некоторой
характеристики, например <<Лучший друг>> или <<Возлюбленный>>.

В случае, если при определении принадлежности лица конкретному человеку 
происходит ошибка, или же принадлежность установлена не была, приложение должно
запросить у пользователя строку, которая характеризует изображенного человека.
После этого должно произойти сохранение набора параметров, характеризующих
лицо, а так же характеристики, введенной пользователем в память для того, чтобы
в будущем программа могла предпринять более успешную попытку определения
принадлежности лица.

\subsubsection{Описание входных и выходных данных}

Как входными, так и выходными данными продукта должен являться фотосник,
полученный с камеры мобильного телефона. При этом, в целях первоначальной
настройки и отладки приложения необходимо допустить возможность его запуска
на настольной операционной системе или же на эмуляторе ОС Android.
При этом, необходимо предусмотреть возможность загрузки изображения из
памяти.

При работе приложения должны так же использоваться некоторые внешние данные,
такие как файл базы данных или сохраненные характеристики уже рассмотренных лиц.
Эти данные должны храниться локально для каждого устройства и не должны быть
получены или переданы каким либо образом, например через интернет.

\subsubsection{Описание требований к вычислительной системе}

Описание системных требований для работы с программой
можно представить следующими пунктами:
\begin{itemize}
    \item Конечная платформа для работы с программным продуктом. \\
        Программа должна работать с одной из платформ семейства Android вервии
        4 на базе процессоров семейства x86, armv7 или armv5.
    \item Необходимые для работы внешние библиотеки. \\
        При работе приложение должна использовать библиотеки Qt и SQLite.
        Библиотека Qt может быть загружена с мобильного устройства в
        автоматизированном режиме при открытии приложения
        приложением Ministro\footnote{%
            Ссылка для скачивания: %
            \url{https://play.google.com/store/apps/details?id=org.kde.necessitas.ministro&hl=ru}}.
            Библиотека SQLite входит в состав ОС Android.
\end{itemize}


\clearpage
