\section{Введение}

Автоматический анализ лиц, который включает в себя обнаружение лиц,
распознавание лиц, распознавание эмоций на лицах, в настоящее время является активно развивающейся областью машинного зрения. Коммерческие,
охранные и другие анализирующие приложения все чаще используют технологию распознавания лиц для сбора информации. Лицо человека является
одним из его главных идентификаторов. Очевидно, что лицо - наиболее видимая часть человеческого тела в обыденной жизни. По нему можно
определить личность человека, а также попробовать определить его эмоциональный настрой.

Каждый из нас имеет уникальные черты лица, которые являются одной из достоверных отличительных особенностей. Распознавание
лиц производится с целью сопоставления лица на входном изображению кластеру изображений лиц в базе данных с некоей точностью. Данный кластер
может состоять как из изображения одного лица, так и из нескольких изображений, принадлежащих одному человеку, что в свою очередь увеличивает
точность распознавания. Актуальность проблемы подтверждает заинтересованность в ней такого технологического гиганта, как Facebook. Например,
согласно \cite{facebook} им удалось вплотную приблизиться к точности распознавания,
сравнимой с человеческой.

С ростом популярности мобильных устройств, увеличивается необходимость в повышении безопасности при доступе к мобильному устройству.
Одним из методов идентификации личности пользователя устройства, является идентификация по внешним признакам.

Целью работы является улучшение показателей распознавания лиц в реальном времени и практическое применение
разработанных алгоритмов в мобильном устройстве.
Среди таких показателей можно выделить скорость работы программы, точность распознавания и количество оперативной памяти,
требуемой при распознавании.

Для достижения поставленной цели необходимо рассмотреть из каких
этапов состоит распознавание лиц и улучшить те этапы, которые сильнее
всего влияют на вышеуказанные показатели. Согласно \cite{facebook} распознавание лиц
состоит из: обнаружения, выравнивания, представления информации о лице
в виде, пригодной для распознавания и, собственно, распознавания. Данная
работа посвящена решению задачи поиска лица на фотоснимке, представлению информации о лице
и последующей классификации.

Методы обнаружения лица должны давать как можно более однозначные результаты, имея при этом
минимальную погрешность. Также важными являются факторы скорости работы,
затрат оперативной памяти и возможность распараллеливания. Алгоритмы обнаружения и
классификации должны работать в реальном времени с низким числом ложных срабатываний.

В аналитическом разделе выполняется обзор предметной области. Рассматриваются существующие методы решения проблем, сравниваются
результаты их работы. Обосновывается выбор настоящего решения. В конструкторском разделе описываются используемые методы или алгоритмы.
Также в нем описываются выбранные способы тестирования, результаты тестирования и структура разрабатываемого программного обеспечения.
Технологический раздел содержит обоснованный выбор средств программной
реализации, описание основных моментов программной реализации и методики
тестирования созданного программного обеспечения. В нем же описывается информация,
необходимая для сборки и запуска разработанного программного обеспечения. В экспериментальном разделе содержится описание
планирования экспериментов и их результаты.

\clearpage
