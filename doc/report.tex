\documentclass[a4paper]{article}

\usepackage[12pt]{extsizes}

\usepackage{mathtext}

\usepackage[T2A]{fontenc}
\usepackage[utf8]{inputenc}

\usepackage[english,russian]{babel}

\usepackage{indentfirst}
\usepackage{scrextend}
\usepackage{fancyhdr}
\newcommand*\rot{\rotatebox{90}}

\usepackage{xr}
\usepackage{verbatimbox}
\usepackage{caption}
\usepackage{graphicx}

\usepackage{gensymb}
\usepackage{tabu}

\usepackage{listings}
\usepackage{color}
\definecolor{mygreen}{rgb}{0,0.6,0}
\definecolor{mygray}{rgb}{0.5,0.5,0.5}
\lstset{extendedchars=\true,
		breaklines=true,
		breakatwhitespace=true,
		captionpos=b,
		keepspaces=true,
		keywordstyle=\color{blue},
		numbers=left,
		tabsize=4,
		numberstyle=\tiny\color{mygray},
		commentstyle=\color{mygreen},
		language=C,
		basicstyle=\ttfamily,
		showstringspaces=false,
		title=\lstname
		}
 
\usepackage{geometry} % Меняем поля страницы
\geometry{left=2.8cm}% левое поле
\geometry{right=2.8cm}% правое поле
\geometry{top=2.5cm}% верхнее поле
\geometry{bottom=3cm}% нижнее поле

\usepackage{perpage}
\usepackage{placeins} % for float barriers
\usepackage{float}
\MakePerPage{footnote}

\setcounter{tocdepth}{2}
\ifx
Выставление глубины оглавления
n=4 это chapter, section, subsection, subsubsection и paragraph;
n=3 это chapter, section, subsection и subsubsection;
n=2 это chapter, section, и subsection;
n=1 это chapter и section;
n=0 это chapter.
\fi

\renewcommand*\thesection{\arabic{section}.}
\renewcommand{\thesubsection}{\thesection\arabic{subsection}}
\renewcommand{\thetable}{\thesection\arabic{table}}
\renewcommand{\theequation}{\thesection\arabic{equation}}

\newcommand{\newpar}{\par\medskip}
\usepackage{multirow,tabularx}

\newcommand{\sbt}{
	\,\begin{picture}(-1,1)(-1,-3)\circle*{4}\end{picture}\ 
}
\newcommand{\sbti}{\sbt~~}

\newcommand{\linux}{Linux}
\newcommand{\linuxv}{\linux\ v.3.9.2\ }
\newcommand{\gnu}{GNU}
\newcommand{\gnulinux}{\gnu/\linux}
\newcommand{\archlinux}{Arch Linux}
\newcommand{\fulllinux}{\archlinux, ядро \linuxv}
\newcommand{\unix}{UNIX\ }

\newcommand{\linuxpath}[1]{\texttt{#1}}
\newcommand{\linuxutil}[1]{\texttt{#1}}
\newcommand{\linuxcommand}[1]{\texttt{#1}}
\newcommand{\src}[1]{\linuxcommand{#1}}
\newcommand{\key}[1]{\src{<#1>}}

\begin{document}
	\thispagestyle{fancy}

\fancyhead[C]{
	Федеральное государственное бюджетное 
	образовательное учреждение высшего 
	профессионального образования\\
	<<Московский государственный технический 
	университет им.~Н.\,Э.~Баумана>>
}
\fancyfoot[C]{ Москва, 2014г. }

\vspace*{2cm}

\begin{flushright}
	\Large{ Факультет: }\\
	\large{ <<Информатика и системы управления>> }\\
	\Large{ Кафедра: }\\
	\large{ <<Программное обеспечение ЭВМ и\\ 
		информационные технологии>> }
\end{flushright}

\vspace{2cm}

\begin{LARGE} 
	\begin{center} 
		Расчетно\,--\,пояснительная записка\\
		к дипломному проекту по теме\\ 
		\vspace{2cm}
        Распознавание человеческого лица по фотоснимку на мобильной
        платформе под управлением ОС Android
	\end{center}
\end{LARGE}

\vspace{5cm}

\begin{flushright}
	\begin{tabular}{ll}
	Руководитель дипломного проекта:&Майков~К.\,А.\\
	Исполнитель дипломного проекта:&Бережной~П.\,Ю.
	\end{tabular}
\end{flushright}

\newpage
\setcounter{page}{1}

	\tableofcontents
    \newpage
    \section{Организационно-экономическая часть}

В данном разделе будет проведен расчет трудоемкости выполнения работ, расчет количества исполнителей,
построение календарного плана, расчет конечной стоимости и экономической эффективности программного продукта (ПП).

\subsection{Введение}

Заказчиком (кафедрой) сформулировано техническое задание на разработку ПП.

Необходимо разработать ПП, осуществляющий распознавание лиц в реальном
времени с приемлемой точностью.

Данный проект выполнен в среде Emacs.
Заказчик предполагает реализовывать ПП на электронном информационном
носителе (компакт-диске).

На рынке подобные продукты представлены как
бесплатными продуктами (OpenCv, PCL), так и коммерческими продуктами (Синезис).

Расчёты производились в соответствии с \cite{econ_1}.

\subsubsection{Организация и планирование процесса разработки}

При использовании традиционного подхода, организация и планирование
процесса разработки программного продукта или программного комплекса
предусматривает выполнение следующих работ:
\begin{itemize}
	\item формирование состава выполняемых работ и группировка их по
	стадиям разработки;
	\item расчет трудоемкости выполнения работ;
	\item установление профессионального состава и расчет количества
	исполнителей;
	\item определение продолжительности выполнения отдельных этапов разработки;
	\item построение календарного графика выполнения разработки;
\end{itemize}

Планирование длительности этапов и содержания проекта осуществляется в
соответствии с ЕСПД ГОСТ 34.603-92 и распределяет работы по этапам,
как показано в таблице~\ref{table:econ_1}.
\begin{table}[ht]
    \centering
	\begin{tabu}[\textwidth]{|c|c|l|}
		\hline
		Основные стадии & № & Содержание работы \\
		\hline
		\multirow{2}{*}{Техническое задание}
			& 1 & Постановка задачи\\
			\cline{2-3}
			& 2 & Выбор средств разработки и реализации\\
		\hline
		\multirow{2}{*}{Эскизный проект}
			& 3 & Разработка математической модели\\
			\cline{2-3}
			& 4 & Разработка алгоритмов расчёта задачи\\
		\hline
		\multirow{3}{*}{Техно-рабочий проект}
			& 5 & Реализация алгоритмов расчёта задачи\\
			\cline{2-3}
			& 6 & Разработка пользовательского интерфейса\\
			\cline{2-3}
			& 7 & Реализация пользовательского интерфейса\\
		\hline
		Внедрение & 8 & Проведение вычислительных экспериментов\\
		\hline
	\end{tabu}
	\captionsetup{justification=centering}
	\caption{Распределение работ по этапам.}
	\label{table:econ_1}
\end{table}

\subsubsection{Расчёт трудоёмкости выполнения работ}

Трудоемкость разработки программной продукции зависит от ряда факторов,
основными из которых являются следующие:
\begin{itemize}
	\item степень новизны разрабатываемого программного комплекса,
	\item сложность алгоритма его функционирования,
	\item объем используемой информации, вид ее представления и способ
	обработки,
	\item уровень используемого алгоритмического языка программирования
\end{itemize}

Исходные данные расчета приведены в таблице~\ref{table:econ_2}
\begin{table}[ht]
    \centering
	\begin{tabu}[\textwidth]{|X[l]|X[l]|}
		\hline
		Функциональное назначение ПП & Управление лицензиями на
		программное обеспечение в техническом университете.\\
		\hline
		Степень новизны разрабатываемого проекта & Группа новизны
		\texttt{B} - разработка программной продукции, имеющей
		аналоги.\\
		\hline
		Степень сложности алгоритма функционирования & \texttt{3}
		группа сложности - программная продукция, реализующая
		алгоритмы стандартных методов решения задач. \\
		\hline
		По виду представления исходной информации & Группа \texttt{12}
		- исходная информация представлена в форме докуметов, имеющих
		одинаковый формат и структуру, требуется форматный контроль
		информации. \\
		\hline
		Структура выходным документов & Группа \texttt{22} - требуется
		вывод на печать одинаковых документов, вывод информационных
		массивов на машинные носители.\\
		\hline
	\end{tabu}
	\captionsetup{justification=centering}
	\caption{Исходные данные}
	\label{table:econ_2}
\end{table}

Трудоемкость разработки программной продукции $\tau_{ПП}$ может быть определена
как сумма величин трудоемкости выполнения отдельных стадий разработки
ПП из выражения:
\begin{equation}
	\tau_{ПП} = \tau_{ТЗ} + \tau_{ЭП} + \tau_{ТП} + \tau_{РП} + \tau_{В}
\label{F:econ_1}
\end{equation}
где
\begin{itemize}
	\item $\tau_{ТЗ}$ - трудоемкость разработки технического задания на
	создание ПП;
	\item $\tau_{ЭП}$ - трудоемкость разработки эскизного проекта ПП;
	\item $\tau_{ТП}$ - трудоемкость разработки технического проекта ПП;
	\item $\tau_{РП}$ - трудоемкость разработки рабочего проекта ПП;
	\item $\tau_{В}$ - трудоемкость внедрения разработанного ПП.
\end{itemize}

Трудоемкость разработки технического задания рассчитывается по формуле:
\begin{equation}
	\tau_{ТЗ} = T_{ЗРЗ} + T_{ЗРП}
\label{F:econ_2}
\end{equation}
где
\begin{itemize}
	\item $T_{ЗРЗ}$ - затраты времени разработчика постановки задач на
		разработку ТЗ, чел. дни;
	\item $T_{ЗРП}$ - затраты времени разработчика программного
		обеспечения на разработку ТЗ, чел. дни.
\end{itemize}

В расчёте участвуют следующие коэффициенты:
\begin{itemize}
	\item $t_{З} = 47$ -- норма времени на разработку ТЗ на программный
		продукт в зависимости от функционального назначения и степени новизны
		разрабатываемого ПП, чел. дни;
	\item $K_{ЗРЗ} = 0,65$ -- коэффициент, учитывающий удельный вес
		трудоемкости работ, выполняемых разработчиком постановки на стадии ТЗ;
	\item $K_{ЗРП} = 0,35$ -- коэффициент, учитывающий удельный вес
		трудоемкости работ, выполняемых разработчиком программного обеспечения
		на стадии ТЗ.
\end{itemize}

Тогда
\begin{equation}
	\tau_{ТЗ} = 47 \cdot (0,65 + 0,35) = 47 [чел. дни]
\label{F:econ_3}
\end{equation}

Аналогично рассчитывается трудоёмкость эскизного проекта ПП $\tau_{ЭП}$:
\begin{equation}
	\tau_{ЭП} = t_{ЭП} \cdot (K_{ЭРЗ} + K_{ЭРП}) = 67 \cdot (0,75 + 0,25) = 67 [чел. дни]
\label{F:econ_4}
\end{equation}

Трудоемкость разработки технического проекта $\tau_{ТП}$ зависит от функционального
назначения ПП, количества разновидностей форм входной и выходной информации
и определяется как сумма времени, затраченного разработчиком постановки
задач и разработчиком программного обеспечения, т.е.
$$\tau_{ТП} = (t_{ТРЗ} + t_{ТРП}) \cdot К_{В} \cdot К_{р}$$
$$К_{В} = (К_{П} n_{П} + К_{НС} n_{НС} + К_{Б} n_{Б} )/(n_{П} + n_{НС} + n_{Б})$$
где
\begin{itemize}
	\item $t_{ТРЗ} = 57, t_{ТРП} = 43$ - норма времени, затрачиваемого на
		разработку технического проекта разработчиком постановки задач и
		разработчиком ПП соответственно, чел.-дни
	\item $K_{p} = 1,26$ - коэффициент учета режима обработки информации
	\item $К_{П} = 1, К_{Н}С = 0,72, К_{Б} = 2, 18$ - значения
		коэффициентов учета вида используемой информации для переменной,
		нормативно-справочной информации и баз данных соответственно
	\item $n_{П} = 6, n_{Н}С = 4, n_{Б} = 0$ - значения коэффициентов
		учета вида используемой информации для переменной,
		нормативно-справочной информации и баз данных соответственно
\end{itemize}

Тогда
$$\tau_{ТП} = (57 + 43)(1 \cdot 6 + 0,72 \cdot 4 + 2,18 \cdot 0)/(6 + 4 + 0)
\cdot 1,26 = 112 [чел. дни]$$

Трудоемкость разработки рабочего проекта $\tau_{РП}$ зависит от функционального
назначения ПП, количества разновидностей форм входной и выходной информации,
сложности алгоритма функционирования, сложности контроля информации,
степени использования готовых программных модулей, уровня алгоритмического
языка программирования и определяется по формуле:
$$\tau_{РП} = К_{к} К_{р} К_{Я} К_{З} К_{ИА} (t_{РРЗ} + t_{РРП})$$
$$K_{ИА} = (К^{'}_{П} \cdot n_{П} + К^{'}_{НС} \cdot n_{НС} + К^{'}_{Б} \cdot
n_{Б})/(n_{П} + n_{НС} + n_{Б} )$$

\begin{itemize}
	\item $t_{РРЗ} = 138, t_{РРП} = 979$ - норма времени, затраченного на
		разработку РП на алгоритмическом языке высокого уровня разработчиком
		постановки задач и разработчиком программного обеспечения соответственно,
		чел.дни.
	\item $K_{K} = 1$ - коэффициент учета сложности контроля информации;
	\item $K_{P} = 1,32$ - коэффициент учета режима обработки информации
	\item $K_{Я} = 1$ - коэффициент учета уровня используемого
		алгоритмического языка программирования;
	\item $K_{З} = 0,8$ - коэффициент учета степени использования готовых
		программных модулей;
	\item $K_{ИА}$ - коэффициент учета вида используемой информации и
		сложности алгоритма ПП;
	\item $К^{'}_{П} = 1,20, К_{Н}С^{'} = 0,65, К^{'}_{Б} = 0,54$ -
		значения коэффициентов учета сложности алгоритма ПП и вида
		используемой информации для переменной, нормативно-справочной
		информации и баз данных соответственно.
\end{itemize}

Тогда
$$К_{ИА} = 6 \cdot 1,20 + 4 \cdot 0,65/(4 + 6) = 0,98$$
$$\tau_{РП} = 1 \cdot 1,32 \cdot 1 \cdot 0,8 \cdot 0,98 \cdot (138 + 979) = 1117 [чел. дни]$$

Так как при разработке ПП стадии «Технический проект» и «Рабочий
проект» объединены в стадию «Техно-рабочий проект», то трудоемкость ее
выполнения $\tau_{ТРП}$ определяется по формуле
$$\tau_{ТРП} = 0,85\tau_{ТП} + \tau_{РП} = 0,85 \cdot 112 + 1117 = 1212 [чел. дни]$$

Трудоемкость выполнения стадии внедрения $\tau_{В}$ может быть рассчитана
по формуле:
$$\tau_{В} = К_{к} К_{р} К_{З} (t_{ВРЗ} + t_{ВРП} ) = 1 \cdot 1,32 \cdot 0,8(33 + 98) = 138 [чел.дни]$$

Трудоемкости по этапам разработки проекта представлены в
таблице~\ref{table:econ_3}.
\begin{table}[ht]
    \centering
	\begin{tabu}[\textwidth]{|X[c]|X[c]|}
		\hline
		Этап & Трудоемкость этапа, [чел. дни]\\
		\hline
		ТЗ & 47\\
		\hline
		ЭП & 67\\
		\hline
		ТРП& 1212\\
		\hline
		В & 138\\
		\hline
		Итого & 1464\\
		\hline
	\end{tabu}
	\captionsetup{justification=centering}
	\caption{Трудоемкости по стадиям разработки проекта}
	\label{table:econ_3}
\end{table}

Средняя численность исполнителей при реализации проекта разработки Q
и внедрения ПО определяется соотношением $N = \frac{Q_{p}}{F}$, где
\begin{itemize}
	\item $Q_{p} = \tau \cdot t_{p}$ - затраты труда на выполнение проекта
		(разработка и внедрение ПО),
	\item $F = T \cdot F_{M}$ - фонд рабочего времени;
	\item $Т$ - время выполнения проекта в месяцах. T = 5 мес.;
	\item $F_{M}$ - фонд времени в текущем месяце, который рассчитывается
		из учета общества числа дней в году, числа выходных и праздничных дней
		и определяется соотношением $F_{M} =
		\frac{t_{p}(D_{k}-D_{B}-D{П}}{12}$
	\item $t_{p}$ - продолжительность рабочего дня;
	\item $D_{K}$ - общее число дней в году;
	\item $D_{B}$ - число выходных дней в году;
	\item $D_{П}$ - число праздничных дней в году.
\end{itemize}

Тогда
$$F = 5 \cdot 8(365 - 103 - 13)/12 = 830$$
$$N = 1464 \cdot 8/830 = 14 - число \quad исполнителей\quad проекта.$$


\subsubsection{Календарный план-график}

Планирование и контроль хода выполнения разработки проводится по
календарному графику выполнения работ. Планирование процесса разработки и
календарный ленточный план представлены в таблице~\ref{table:econ_4}
и рис.~\ref{pic:econ_1} соответственно.

Вывод: при распараллеливании работы ведущего инженера и программи-
стов можно добиться сокращения срока разработки и внедрения программ-
ного продукта с 1464 дней до 211 дней, т. е. в 6,94 раза по сравнению со
временем разработки одним человеком.

В таблице~\ref{table:econ_5} приведены затраты на заработную плату и
отчисления на социальное страхование в пенсионный фонд, фонд
занятости и фонд обязательного медицинского страхования (30\%). Для
всех исполнителей предполагается оклад в размере 20000 рублей в месяц.

Расходы на материалы, необходимые для разработки программной продукции, указаны в таблице~\ref{table:econ_6}.

В работе над проектом используется специальное оборудование – персональные
электронно-вычислительные машины (ПЭВМ) в количестве 14 шт.
Стоимость одной ПЭВМ составляет 20 000 рублей. Согласно нормативным
документам, срок амортизации ПЭВМ составляет 3 года, что определяет
месячную норму амортизации K = 2,7\%.

Тогда за 10 месяцев работы расходы на амортизацию составят $20 000 \cdot 14\cdot
0,027 \cdot 10 = 75 600руб.$

Общие затраты на разработку ПП составят:
$$C = 728 000 + 1 900 + 75 600 = 805 500 руб.$$

\begin{table}
    \centering
	\begin{tabu}[\textwidth]{|X[l]|X[l]|X[c]|X[l]|X[l]|}
	\hline
	Стадия & $\tau$ & Должность исполнителя & Распределение трудоемкости & Ч-ть \\
	\hline
	\multirow{2}{*}{ТЗ} & 47 & Ведущий инженер & 37(79\%) & 1 \\
			& & Программист & 10 & 1 \\
	\hline
	\multirow{2}{*}{ЭП} & 67 & Ведущий инженер & 37(55\%) & 1 \\
			& & Программист & 30 & 1 \\
	\hline
	\multirow{2}{*}{ТРП} & 1212 & Ведущий инженер & 81 & 1 \\
				& & Программист & 13 х 87 & 13 \\
	\hline
	\multirow{2}{*}{B} & 138 & Ведущий инженер & 38 & 1 \\
			 & & Программист & 2 x 50 & 2 \\
	\hline
	Итого & 1464 & & & 14 \\
	\hline
	\end{tabu}
	\captionsetup{justification=centering}
	\caption{Планирование процесса разработки.}
	\label{table:econ_4}
\end{table}

\begin{figure}[ht]
	\centering
	\includegraphics{econ_1.png}
	\caption{Календарный ленточный план работ.}
	\label{pic:econ_1}
\end{figure}

\begin{table}
    \centering
	\begin{tabularx}{\textwidth}{|c||X|X|X||X|X|X||X|X|X||X|X|X||c|}
		\hline
		& \multicolumn{3}{c||}{Г.И.} & \multicolumn{3}{c||}{П1} &
		\multicolumn{3}{c||}{П2} & \multicolumn{3}{c||}{П3..14} & Всего\\
		\hline
		Месяц & Р.Д. & ЗП & ЕСН & Р.Д. & ЗП & ЕСН & Р.Д & ЗП & ЕСН &
		Р.Д. & ЗП & ЕСН & за период\\
		\hline
		1 & 21 & 20 & 6 & 10 & 9.52 & 2.85 & & & & & & & 38.38\\
		\hline
		2 & 21 & 20 & 6 & 5 & 4.76 & 1.42 & & & & & & & 32.19\\
		\hline
		3 & 21 & 20 & 6 & 21 & 20 & 6 & & & & & & & 52\\
		\hline
		4 & 21 & 20 & 6 & 14 & 13.33 & 4 & 7 & 6.66 & 2 & 7 & 6.66 & 2 & 60.67\\
		\hline
		5 & 21 & 20 & 6 & 21 & 20 & 6 & 21 & 20 & 6 & 21 & 20 & 6 & 104.00\\
		\hline
		6 & 21 & 20 & 6 & 21 & 20 & 6 & 21 & 20 & 6 & 21 & 20 & 6 & 104.00\\
		\hline
		7 & 21 & 20 & 6 & 21 & 20 & 6 & 21 & 20 & 6 & 21 & 20 & 6 & 104.00\\
		\hline
		8 & 15 & 14.28 & 4.28 & 21 & 20 & 6 & 21 & 20 & 6 & 14 & 13.33 & 4 & 87.90\\
		\hline
		9 & 21 & 20 & 6 & 21 & 20 & 6 & 21 & 20 & 6 & & & & 78.00\\
		\hline
		10 & 10 & 9.52 & 2.85 & 22 & 20.95 & 6.28 & 22 & 20.95 & 6.28 & & & & 66.86\\
		\hline
		\multicolumn{13}{|l||}{Итого: } & 728.00 \\
		\hline
	\end{tabularx}
	\captionsetup{justification=centering}
	\caption{Затраты на зарплату и отчисления на социальное страхование, тыс.руб.}
	\label{table:econ_5}
\end{table}

\begin{table}[ht]
    \centering
	\begin{tabu}[\textwidth]{|X[c]|X[c]|X[c]|X[c]|X[c]|}
	\hline
	Наименование материала & Единица измерения & К-во & Цена/ед. (руб.) & Сумма (руб.) \\
	\hline
	Персональный компьютер & ASUS K52-JT & 14 & 25000 & 350000 \\
	\hline
	Офисная мебель & стулья и столы ikea & 14 & 5000 & 70000 \\
	\hline
	Бумага А4 & Пачка 500 листов & 2 & 200 & 400 \\
	\hline
	Картридж принтера Canon IP5200 & Картридж, 10мл & 5 & 300 & 1500 \\
	\hline
	\multicolumn{4}{|l|}{Итого:} & 421900 \\
	\hline
	\end{tabu}
	\captionsetup{justification=centering}
	\caption{Затраты на материалы.}
	\label{table:econ_6}
\end{table}

\subsubsection{Расчёт стоимости программного продукта}

Цена ПП рассчитывается по формуле:
$$Ц = С + Пр$$
$$Пр = \frac{(C - C_{M}) \cdot p_{Н}}{100\%}$$
где
\begin{itemize}
	\item $С$ - затраты на разработку ПП
	\item $С{м}$ - материальные затраты, руб./изд
	\item $Пр$ - желаемая прибыль
	\item $р_{н}$ - норматив рентабельности, принимаемый разработчиком
\end{itemize}

Тогда примем $Ц = 5 000 руб.$

Нужно продать 350 лицензионные копии, чтобы окупить вложенные средства.

\subsection{Рассчет экономической эффективности}

Основными показателями экономической эффективности является чистый
дисконтированный доход (ЧДД) и срок окупаемости вложенных средств.

Чистый дисконтированный доход определяется по формуле:
$$ЧДД = \sum^{T}_{t=0}(R_{t} - З_{t}) \frac{1}{(1 + E)^t}$$
где
\begin{itemize}
	\item $T$ - горизонт расчета по месяцам;
	\item $t$ - период расчета;
	\item $R_{t}$ - доход за текущий месяц;
	\item $З_{t}$ затраты за текущий месяц;
	\item $E$ - приемлемая для инвестора норма прибыли на вложенный капитал.
\end{itemize}

Коэффициент E установим равным ставке рефинансирования ЦБ РФ --
8.25\% годовых (или 0, 66\% в месяц). В виду особенности разрабатываемого
продукта он может быть продан лишь однократно. Коэффициент дисконти-
рования равен 1/(1 + Е) = 0,993.

В таблице~\ref{table:econ_6} приведен расчет ЧДД по месяцам работы над проектом.
График ЧДД приведён на рис.~\ref{pic:econ_2}.

\begin{table}
    \centering
	\begin{tabu}[\textwidth]{|c|c|c|c|c|}
	\hline
	Месяц & Тек. Затр. & Общ. Затр. & Тек.доход & ЧДД \\
	\hline
	1 & 46130,95 & 46130,95 & 0,00 & -45817,25 \\
	\hline
	2 & 39940,48 & 86071,43 & 0,00 & -85216,39 \\
	\hline
	3 & 59750,00 & 145821,43 & 0,00 & -143755,76\\
	\hline
	4 & 68416,67 & 214238,10 & 0,00 & -210330,39\\
	\hline
	5 & 111750,00 & 325988,10 & 0,00 & -318332,22\\
	\hline
	6 & 111750,00 & 437738,10 & 0,00 & -425599,63\\
	\hline
	7 & 111750,00 & 549488,10 & 0,00 & -532137,62\\
	\hline
	8 & 95654,76 & 645142,86 & 0,00 & -622710,94\\
	\hline
	9 & 85750,00 & 730892,86 & 0,00 & -703353,54\\
	\hline
	10 & 74607,14 & 805500,00 & 987500,00 & 149328,11 \\
	\hline
	\end{tabu}
	\captionsetup{justification=centering}
	\caption{Расчёт ЧДД (все значения в руб.).}
	\label{table:econ_6}
\end{table}

\begin{figure}
    \centering
	\includegraphics{econ_2.png}
	\caption{График изменения чистого дисконтированного дохода.}
	\label{pic:econ_2}
\end{figure}

\subsection*{Выводы}
\addcontentsline{toc}{subsection}{Выводы}

Согласно проведённым расчётам, проект является рентабельным. Итоговый
ЧДД составил $149328.11$ рублей. Срок реализации проекта равен 10 месяцам.

Поскольку программный продукт - рентабелен, он окупится после выхода на рынок.
Учитывая курс правительства РФ на модернизацию науки и техники, следует
ожидать повышенный спрос не только на конкретную реализацию, а также на
научный труд, подкрепляющий данную реализацию. Также данный программный
продукт может быть использован в широком спектре ниш: распознавание в метро,
при входе на предприятие, автоматическая блокировка ПК, домашние двери.
Это позволяет предположить, что программный продукт не только окупится,
но и принесет ощутимую прибыль при его должном развитии.
\clearpage

	\newpage
    \section{Условия труда}
В настоящем разделе будут рассмотрены условия, в которых
находился работник при разработке программного обеспечения.

\subsection{Анализ опасных и вредных факторов при разработке программного
    обеспечения и мероприятия по их устранению}

Разработка программного обеспечения требует постоянного взаимодействия с
вычислительными машинами, что связано с воздействием ряда вредных и зачастую опасных факторов, таких
как статическое электричество, рентгеновское излучение, электромагнитные поля,
ультрафиолетовое излучение, блики, отраженный свет и мерцание изображения.
Рассмотрим более подробно некоторые из вышеуказанных факторов.

\subsubsection{Микроклимат}

Работа за компьютером не требует серьезных физических усилий, поэтому ее относят к категории \textit{1а}.
Оптимальные нормы микроклимата для этой категории определяются таблицей \textit{СанПиН 2.2.2/2.4.1340-03}
(таблица~\ref{tables:microclimate}).

\begin{table}[hbt!]
\begin{tabu}[\textwidth]{|X[c]|X[c]|X[c]|X[c]|}
    \hline
    & Температура воздуха,~\celsius & Относительная влажность воздуха,~\% & Скорость движения воздуха, $ м/с $ \\
    \hline
    Холодный & 22-24 & 40-60 & 0.1 \\
    \hline
    Теплый & 23-25 & 40-60 & 0.1 \\
    \hline
\end{tabu}
\caption{Оптимальные нормы микроклимата}
\label{tables:microclimate}
\end{table}

Вредным фактором при работе с ЭВМ является также запыленность помещения. Этот фактор усугубляется влиянием на частицы пыли
электростатических полей персональных компьютеров.

Для устранения несоответствия параметров указанным нормам проектом предусмотрено использование системы кондиционирования как
наиболее эффективного и автоматически функционирующего средства.

Нормы \textit{СанПиН 2.2.4.1294-03} <<Санитарно-гигиенические нормы допустимых уровней ионизации воздуха>> определяют уровни положительных и
отрицательных ионов в воздухе (таблица~\ref{tables:gigenic}):

\begin{table}[hbt!]
\begin{tabu}[\textwidth]{|X[c]|X[c]|X[c]|}
    \hline
    \multirow{2}{*}{Уровни} & \multicolumn{2}{c|}{Число ионов в 1 см куб. воздуха} \\
    \cline{2-3}
    & $ n^+ $ & $ n^- $ \\
    \hline
    Минимально необходимые & 400 & 600 \\
    \hline
    Оптимальные & 1500-3000 & 3000-5000 \\
    \hline
    Предельно допустимые & 50000 & 50000 \\
    \hline
\end{tabu}
\caption{Уровни ионизации воздуха помещений при работе на ВДТ и ПЭВМ}
\label{tables:gigenic}
\end{table}

Для обеспечения требуемых уровней предусмотрено использование системы ионизации \textit{Сапфир-4А}.

Концентрация вредных химических веществ в помещениях с ПЭВМ не должна превышать <<ПДК загрязняющих веществ в атмосферном воздухе
населенных мест>> \textit{ГН 2.1.6.789-99}. Для выполнения указанных требований предусмотрено применение фильтров из активированного угля.

\subsubsection{Шум и вибрации}

Уровень шума на рабочем месте программиста не должен превышать 50 дБА, а уровень вибрации не должен превышать допустимых норм
вибрации. \textit{СанПиН 2.2.2.542-96} устанавливает следующие нормы на вибрацию (таблица \ref{tables:vibration}).
\begin{table}[hbt!]
\begin{tabu}[0.8\textwidth]{|X[c]|X[c]|X[c]|}
    \hline
    Среднегеометрические & \multicolumn{2}{c|}{Допустимые значения} \\
    частоты октавных полос, Гц & \multicolumn{2}{c|}{по виброскорости} \\
    \cline{2-3}
    & $ \times 10, м/с $ & $ дБ $ \\
    \hline
    2 & 4.5 & 79 \\
    \hline
    4 & 2.2 & 73 \\
    \hline
    8 & 1.1 & 67 \\
    \hline
    16 & 1.1 & 67 \\
    \hline
    31.5 & 1.1 & 67 \\
    \hline
    63 & 1.1 & 67 \\
    \hline
    Корректированные значения и их уровни & 2.0 & 72 \\
    \hline

\end{tabu}
\caption{Допустимые нормы вибрации на рабочих местах с ВДТ и ПЭВМ}
\label{tables:vibration}
\end{table}

При разработке программного обеспечения внутренними источниками шума являются вентиляторы, а также принтеры и другие периферийные
устройства ЭВМ. 

Внешние источники шума -- прежде всего, шум с улицы и из соседних помещений. Постоянные внешние источники шума, превышающего нормы,
отсутствуют.

Для устранения превышения нормы проектом предусмотрено применение звукопоглощающих материалов для облицовки стен и потолка
помещения, в котором осуществляется работа с вычислительной техникой.

\subsubsection{Освещение}
Наиболее важным условием эффективной работы программистов и пользователей является соблюдение оптимальных параметров системы
освещения в рабочих помещениях.

Естественное освещение осуществляется через светопроемы, ориентированные в основном на север и северо-восток (для исключения
        попадания прямых солнечных лучей на экраны компьютеров) и обеспечивает коэффициент естественной освещенности (КЕО) не ниже
1,5\%.

В качестве искусственного освещения проектом предусмотрено использование системы общего равномерного освещения. В соответствии с
\textit{СанПиН 2.2.2/2.4.1340-03},  освещенность на поверхности рабочего стола находится в пределах 300-500 лк. Разрешается использование
светильников местного освещения для работы с документами (при этом светильники не должны создавать блики на поверхности экрана).

Правильное расположение рабочих мест относительно источников освещения, отсутствие зеркальных поверхностей и использование матовых
материалов ограничивает прямую (от источников освещения) и отраженную (от рабочих поверхностей) блескость. При этом яркость
светящихся поверхностей не превышает 200 кд/кв.м, яркость бликов на экране ПЭВМ не превышает 40 кд/кв.м, и яркость потолка не
превышает 200 кд/кв.м.

В соответствии с \textit{СанПиН 2.2.2/2.4.1340-03}  проектом предусмотрено использование люминесцентных ламп типа ЛБ в качестве источников
света при искусственном освещении. В светильниках местного освещения допускается применение ламп накаливания.

Применение газоразрядных ламп в светильниках общего и местного освещения обеспечивает коэффициент пульсации не более 5\%.
Таким образом, проектом обеспечиваются оптимальные условия освещения рабочего помещения.

\subsubsection{Рентгеновское излучение}
В соответствии с \textit{СанПиН 2.2.2/2.4.1340-03}, проектом предусмотрено использование ПЭВМ, конструкция которых обеспечивает мощность
экспозиционной дозы рентгеновского излучения в любой точке на расстоянии 0,05 м. от экрана и корпуса монитора не более 0,1
мбэр/час (100 мкР/час). Результаты сравнения норм излучения приведены в таблице \ref{tables:rentgen}.

\begin{table}[hbt!]
\begin{tabu}[\textwidth]{|X[c]|X[c]|}
    \hline
    & Допустимое значение, $ мкР/час $, не более \\
    \hline
    СанПиН 2.2.2/2.4.1340-03 & 100 \\
    \hline
    TCO-99 & 500 \\
    \hline
    MPR II & 500 \\
    \hline
\end{tabu}
\caption{Сравнение норм рентгеновского излучения в различных стандартах}
\label{tables:rentgen}
\end{table}

Как видно из таблицы \ref{tables:rentgen}, стандарты \textit{MPR II} и \textit{TCO-99} предъявляют менее жесткие требования к
рентгеновскому излучению, чем СанПиН. Но
при соблюдении оптимального расстояния между пользователем и монитором дозы рентгеновского излучения не опасны для большинства
людей.

\subsubsection{Неионизирующие электромагнитные излучения}

В соответствии с \textit{санпин 2.2.2/2.4.1340-03}, допустимые значения параметров неионизирующих излучений приводятся в таблицах
    \ref{tables:electric} и \ref{tables:magnet}.
\begin{table}[Hbt!]
\begin{tabu}[\textwidth]{|X[c]|X[c]|}
    \hline
    Диапазон частот, $ \times 10^3 $, гц & Допустимые значения, $В/м$ \\
    \hline
    $ 5 \times 10^{-3}$ --- 2 & 25 \\
    \hline
    2 --- 400 & 2.5 \\
    \hline
\end{tabu}
\caption{Предельно допустимые значения напряженности электрического поля}
\label{tables:electric}
\end{table}

\begin{table}[Hbt!]
\begin{tabu}[\textwidth]{|X[c]|X[c]|}
    \hline
    Диапазон частот, $ \times 10^3 $, гц & Допустимые значения, $нТл$ \\
    \hline
    $ 5 \times 10^{-3}$ --- 2 & 250 \\
    \hline
    2 --- 400 & 25 \\
    \hline
\end{tabu}
\caption{Предельно допустимые значения плотности магнитного потока}
\label{tables:magnet}
\end{table}

Величина поверхностного электростатического потенциала не должна превышать 500 В. 

Мониторы, используемые в настоящее время, удовлетворяют нормам \textit{MPR II} (или более жестким требованиям) и имеют предельные
значения, указанные в таблицах \ref{tables:electromagnet} и \ref{tables:induction}
\begin{table}[Hbt!]
\begin{tabu}[\textwidth]{|X[c]|X[c]|}
    \hline
    Диапазон частот, $ \times 10^3 $, гц & Допустимые значения, $В/м$ \\
    \hline
    $ 5 \times 10^{-3}$ --- 2 & 25 \\
    \hline
    2 --- 400 & 2.5 \\
    \hline
\end{tabu}
\caption{Предельно допустимые значения напряженности электромагнитного поля}
\label{tables:electromagnet}
\end{table}

\begin{table}[Hbt!]
\begin{tabu}[\textwidth]{|X[c]|X[c]|}
    \hline
    Диапазон частот, $ \times 10^3 $, гц & Допустимые значения, $нТл$ \\
    \hline
    $ 5 \times 10^{-3}$ --- 2 & 200 \\
    \hline
    2 --- 400 & 25 \\
    \hline
\end{tabu}
\caption{Предельно допустимые значения магнитной индукции}
\label{tables:induction}
\end{table}

Поверхностный электростатический потенциал не превышает 500 В.

Таким образом, параметры электрических и магнитных (неионизирующих) полей удовлетворяют требованиям \textit{СанПиН}.

\subsubsection{Визуальные параметры}

Неправильный выбор визуальных эргономических параметров приводит к ухудшению здоровья пользователей, быстрой утомляемости,
раздражительности. В этой связи, проектом предусмотрено, что конструкция вычислительной системы и ее эргономические параметры
обеспечивают комфортное и надежное считывание информации.

Требования к визуальным параметрам, их внешнему виду, дизайну, возможности настройки представлены в \textit{СанПиН 2.2.2/2.4.1340-03}.
Визуальные эргономические параметры монитора и пределы их изменений приведены в таблице \ref{tables:vdt_params}.
\begin{table}[hbt!]
\begin{tabu}[\textwidth]{|X[c]|X[c]|X[c]|}
    \hline
    Наименование & \multicolumn{2}{c|}{Пределы значений параметров} \\
    \cline{2-3}
    параметров & минимальные (не менее) & максимальные (не более) \\
    \hline
    Яркость знака (яркость фона), $ кд/кв.м. $ (измеренная в темноте) & 35 & 120 \\
    \hline
    Внешняя освещенность экрана, $ лк $ & 100 & 250 \\
    \hline
    Угловой размер знака, $ угл.мин. $ & 16 & 60 \\
    \hline
\end{tabu}
\caption{Допустимые нормы вибрации на рабочих местах с ВДТ и ПЭВМ}
\label{tables:vdt_params}
\end{table}

Для выполнения этих требований проектом предусмотрено использование современных мониторов, имеющих достаточно широкий набор
регулируемых параметров. В частности, для удобного считывания информации реализована возможность настройки положения монитора по
горизонтали и вертикали. Мониторы оснащены специальными устройствами и средствами настройки ширины, высоты, яркости, контраста и
разрешения изображения. Кроме того, в современных мониторах зерно изображения имеет размер в пределах 0,27 мм, что обеспечивает
высокую четкость и непрерывность изображения. Наконец, на поверхность дисплея нанесено матовое покрытие, чтобы избавиться от
солнечных бликов.

\subsection{Расчет системы искусственного освещения}

В зависимости от цели расчета при проектировании искусственного освещения приходится решать следующий ряд вопросов:
\begin{itemize}
    \item Выбрать или определить типы ламп и светильников. Для освещения предприятий службы быта следует применять
        газоразрядные лампы. Применение ламп накаливания целесообразно при температуре воздуха ниже 10\celsius и падении напряжения в
            сети более 10\% от номинального.
            Выбор светильника должен производится с учетом его крепления, подвода электроэнергии, защиты от механических
            повреждений, взрыво- и пожароопасности (открытые, закрытые, пылевлагонепроницаемые, взрывоопасные, взрывозащищенные
                    светильники);

    \item Выбрать систему освещения. Наиболее экономичной является система комбинированного освещения, так как она создает
    наиболее равномерное светораспределение.
    При комбинированном освещении доля общего освещения в нем не должна быть меньше 10\%;

    \item 3. Выбрать расположение светильников и определить их количество. Светильники, расположенные симметрично вдоль или
    поперек помещения, в шахматном порядке, рядами, ромбовидно, обеспечивают равномерное по площади освещение. Локализованное
    неравномерное размещение светильников производят с учетом местонахождения ПЭВМ, оборудования и т.д.

    Экспериментально установлено, что наибольшая равномерность достигается:
    \begin{itemize}
        \item При шахматном расположении, если
            $ \frac{r}{H_p} <= 1.7\div 2.5 $
        \item При расположении прямоугольником, если
            $ \frac{r}{H_p} <= 1.4\div 2.0 $
    \end{itemize}
    где $ r $ -- расстояние между светильниками; $ H_p $ -- высота подвеса светильника над рабочей поверхностью:
    \begin{equation}\label{eq:light_height}
        H_p = H - h_c - h_{р.м.}
    \end{equation}
    где $ H $ -- высота помещения; $ h_c $ --  высота подвеса светильника; $ h_{р.м.} $ --
    высота рабочего места ($h_{р.м.} = 0,8 м$).

    Оптимальное расстояние от крайнего ряда светильников до стены:
        $$ r_k = (0.24 \div 0.3) \cdot r $$
    При отсутствии рабочих поверхностей у стены: $$ r_k = (0.4 \div 0.5) \cdot r $$
    Для исключения слепящего действия светильников общего освещения должно выполняться правило
    $$ H - h_c <= 2.5 \div 4\ (м) $$ при мощности ламп $ P_л <= 200 $ Вт.

    Необходимое число светильников при расположении квадратом составляет:
    \begin{equation}\label{eq:lights_count}
        N_c = \frac{S}{r^2}
    \end{equation}
    где $S$ -- площадь помещения; $ r $ -- длина стороны квадрата.

    \item Определить нормируемую освещенность рабочего места по минимальному размеру объекта различия, фону, контрасту объекта
    с фоном в системе освещения.
\end{itemize}

Для расчета искусственного освещения используют три метода:
\begin{itemize}
    \item Метод светового потока для общего равномерного освещения горизонтальной рабочей поверхности;
    \item Точечный метод для любой системы освещения;
    \item Метод удельной мощности для ориентировочных расчетов общего равномерного освещения.
\end{itemize}

Световой поток определяется по формуле:
\begin{equation}\label{eq:light_flux}
F_л = \frac{E_n \cdot K \cdot S \cdot Z}{N \cdot \eta}
\end{equation}
где $F_л$ -- световой поток лампы; $Е_н$ -- нормированная освещенность; $S$ -- площадь освещаемого помещения; $K$ -- коэффициент
запаса (в соответствии со \hbox{\textit{СНиП 23-05-95}} для люминесцентных ламп производственных цехов предприятий службы быта $ K = 1.6 \div 1.7 $; для остальных
        помещений $K = 1.5$); $Z$ -- коэффициент минимальной освещенности, равный отношению средней освещенности к минимальной;
$N$~--~число ламп;
$\eta$ -- коэффициент использования светового потока, равный отношению потока, падающего на рабочую поверхность, к общему
потоку ламп.

Коэффициент использования светового потока $\eta$ зависит от КПД светильника, коэффициента отражения потолка ($ \rho_p $), стен
($\rho_c$),
    величины показателя помещения $i$, учитывающего геометрические параметры помещения, высоту подвеса светильника ($H_p$):
\begin{equation}\label{eq:kpd}
    i = \frac{a \cdot b}{H_p \cdot (a + b)}
\end{equation}
где $ a $ и $ b $ -- ширина и длина помещения.

При длине рабочего помещения $a = 16 м$, ширине $b = 10 м$ и высоте $H = 3.6 м$, потребуется следующее освещение:
$$
    E_н = 400\ лк;\ 
    F_л = 5220\ лм
$$

Тогда:
$$
    i = \frac{a \cdot b}{(H - (h_c + h_{р.м.}))\cdot(a + b)} = \frac{10 \cdot 16}{(3.6 - 0.1 - 0.8) \cdot (10 + 16)} \approx 2.3
$$

Следовательно $$ \eta = 0.41 $$

Из формулы \ref{eq:light_flux} следует, что
$$ N = \frac{N_n \cdot K \cdot S \cdot Z}{F_л \cdot \eta} = \frac{400 \cdot 160 \cdot 1.6 \cdot 1.1}{5220 \cdot 0.41} = 52 $$
что при использовании светильников, состоящих из 3х ламп, потребует 18 светильников.

Расстояние между светильниками равно:
$$
    r = 1.5 \cdot (3.6 - 0.1 - 0.8) = 4\ м
$$

Расстояние от стены до светильников:
$$ r_k = 0.25 \cdot 4 = 1\ м $$
\begin{figure}[htb!]
\centering
\includegraphics[width=0.6\textwidth]{lights}
\caption{Схема освещения помещения}
\label{pic:lights}
\end{figure}

Следовательно, светильники следует расположить в три ряда по шесть светильников, как показано на рисунке \ref{pic:lights}.

При этом имеет место избыточное освещение, превышающее расчетный световой поток на $(54-52) / 52=3.8\%$ (допустимым является 20\%
        отклонение).
Таким образом, в проекте используются $18$ светильников с высотой подвеса $0.1\ м$ и, соответственно, $54$ люминесцентных ламп
\textit{ЛБ-80} со
световым потоком $5220\ лм$ и световой отдачей $65.3\ лм/Вт$.


%	\section{Введение}

Автоматический анализ лиц, который включает в себя обнаружение лиц,
распознавание лиц, распознавание эмоций на лицах, в настоящее время является активно развивающейся областью машинного зрения. Коммерческие,
охранные и другие анализирующие приложения все чаще используют технологию распознавания лиц для сбора информации. Лицо человека является
одним из его главных идентификаторов. Очевидно, что лицо - наиболее видимая часть человеческого тела в обыденной жизни. По нему можно
определить личность человека, а также попробовать определить его эмоциональный настрой.

Каждый из нас имеет уникальные черты лица, которые являются одной из достоверных отличительных особенностей. Распознавание
лиц производится с целью сопоставления лица на входном изображению кластеру изображений лиц в базе данных с некоей точностью. Данный кластер
может состоять как из изображения одного лица, так и из нескольких изображений, принадлежащих одному человеку, что в свою очередь увеличивает
точность распознавания. Актуальность проблемы подтверждает заинтересованность в ней такого технологического гиганта, как Facebook. Например,
согласно \cite{facebook} им удалось вплотную приблизиться к точности распознавания,
сравнимой с человеческой.

С ростом популярности мобильных устройств, увеличивается необходимость в повышении безопасности при доступе к мобильному устройству.
Одним из методов идентификации личности пользователя устройства, является идентификация по внешним признакам.

Целью работы является улучшение показателей распознавания лиц в реальном времени и практическое применение
разработанных алгоритмов в мобильном устройстве.
Среди таких показателей можно выделить скорость работы программы, точность распознавания и количество оперативной памяти,
требуемой при распознавании.

Для достижения поставленной цели необходимо рассмотреть из каких
этапов состоит распознавание лиц и улучшить те этапы, которые сильнее
всего влияют на вышеуказанные показатели. Согласно \cite{facebook} распознавание лиц
состоит из: обнаружения, выравнивания, представления информации о лице
в виде, пригодной для распознавания и, собственно, распознавания. Данная
работа посвящена решению задачи поиска лица на фотоснимке, представлению информации о лице
и последующей классификации.

Методы обнаружения лица должны давать как можно более однозначные результаты, имея при этом
минимальную погрешность. Также важными являются факторы скорости работы,
затрат оперативной памяти и возможность распараллеливания. Алгоритмы обнаружения и
классификации должны работать в реальном времени с низким числом ложных срабатываний.

В аналитическом разделе выполняется обзор предметной области. Рассматриваются существующие методы решения проблем, сравниваются
результаты их работы. Обосновывается выбор настоящего решения. В конструкторском разделе описываются используемые методы или алгоритмы.
Также в нем описываются выбранные способы тестирования, результаты тестирования и структура разрабатываемого программного обеспечения.
Технологический раздел содержит обоснованный выбор средств программной
реализации, описание основных моментов программной реализации и методики
тестирования созданного программного обеспечения. В нем же описывается информация,
необходимая для сборки и запуска разработанного программного обеспечения. В экспериментальном разделе содержится описание
планирования экспериментов и их результаты.

\clearpage
	
%	\input{analizis}
%	\subsection{Конструкторский раздел}

Lalala

\subsection{Метод Виолы-Джонса}

Хотя метод был разработан и представен еще в 2001 году Полом Виолой и Майклом
Джонсом \cite{viola_jones}, он до сих пор является основополагающим для поиска
объектов на изображении в реальном времени.
Алгоритм может распознавать различные классы изображений, однако основной задачей
при его создании было обнаружение лиц.

Существует множество реализаций этого метода, в том числе в составе библиотеки
компьютерного зрения OpenCV \cite{opencv}.

Основные принципы, на которых реализован алгоритм, таковы:
\begin{itemize}
    \item используются изображения в интегральном представлении,
        что позволяет вычислять быстро необходимые объекты;
    \item используются признаки Хаара, с помощью которых происходит
        поиск нужного объекта (в данном контексте, лица и его черт);
    \item используется бустинг (от англ. \textit{boost}~-- улучшение,
        усиление) для выбора наиболее подходящих признаков для искомого
        объекта на данной части изображения;
    \item все признаки поступают на вход классификатора,
        который даёт результат <<верно>> либо <<ложь>>;
    \item используются каскады признаков для быстрого отбрасывания окон,
        где не найдено лицо.
\end{itemize}

Обучение классификаторов идет очень медленно, но результаты поиска
лица очень быстры, именно поэтому был выбран данный метод
распознавания лиц на изображении. Виола-Джонс является
одним из лучших по соотношению показателей эффективность
распознавания/скорость работы. Также этот детектор
обладает крайне низкой вероятностью ложного обнаружения лица. Алгоритм
даже хорошо работает и распознает черты
лица под небольшим углом, примерно до 30 градусов. При угле
наклона больше 30 градусов
процент обнаружений резко падает.

Данный метод в общем виде ищет лица и черты лица по общему
принципу сканирующего окна.

\clearpage

%	\section{Технологический раздел}

В данном разделе будут рассмотрены 
средства программной 
реализации, описание основных моментов программной реализации и 
методики тестирования созданного программного обеспечения. 

\subsection{Технические средства}

В данном разделе будут описаны технические средства, которые
были использованы в процессе разработки программного обеспечения.

\subsubsection{Язык программирования}

Для разработки был выбран язык программирования C++. Из его достоинств следует
выделить высокую скорость (практически сравнимую с языком C для современных
компиляторов) и широкие возможности, позволяющие достаточно писать достаточно
удобный и выразительный код по сравнению с тем же C, а так же вероятно самое
большое количество удобных сторонних библиотек на все случаи жизни. Также C++
предоставляет громоздкие, но крайне эффективные методы шаблонного
метапрограммирования, позволяющие генерировать очень эффективный код из
неповторяющихся блоков. Из недостатков следует выделить относительную сложность синтаксиса и
большое количество правил языка и нетривиальных моментов в языке, из чего
следует потенциально большое количество скрытых ошибок.

Было выбрано подмножество языка C++11, которое даёт разработчику новые
возможности по написанию более быстрого и понятного кода, большие приятные
нововведения в библиотеку C++ STL, а так же новые конструкции для избавления от
некоторых ошибок.

В качестве альтернатив рассматривались некоторые , скриптовые языки общего назначения с
привязками к библиотекам с машинным кодом (Python, Ruby, Perl), языки общего
назначения со сборкой мусора, исполняемые на виртуальной машине (Java,
Scala). Поскольку проект предполагает громоздкие вычисления с необходимостью
расчётов в реальном времени, был сделан выбор в пользу компилируемых в машинный
код языков программирования, среди которых был выбран C++ как наиболее удобный и
функциональный, а также как один из самых быстрых.  На момент начала работы
основные аспекты языка уже были изучены, что позволило целиком
сконцентрироваться на задаче, а не тратить время на изучения инструмента.

\subsubsection{Компилятор}

В процессе разработки использовался компилятор языка C++ GCC.
Он представляет собой самый развитый и популярный свободный
компилятор для языка C++ в мире, включает в себя большое множество оптимизаций,
поддерживаемых платформ и генерирует крайне эффективный код.
Компилятор был выбран по соображениям
большого количества поддерживаемых платформ, в том числе Linux, на котором в
основном разрабатывался данный проект.

\subsubsection{Утилиты и среды разработки}

Проект разрабатывался в нескольких средах разработки. В основном разработка
делилась на два цикла~-- написание кода и отладка. Для написания кода
использовалась среда разработки QtCreator. Эта среда предоставляет
разработчику удобные средства создания кода, включая интуитивную подсветку синтаксиса
библиотеки Qt5 и встроенный отладчик.
Альтернативами приведённым инструментам является текстовый редактор
EMacs и интерфейс отладчика GDB CGDB, предоставляющий консольный
псевдo-оконный интерфейс для работы.

Написание пояснительной записки к проекту производилось в текстовом редакторе Vim с набором расширений, обеспечивающих
предельно комфортную работу с исходным кодом на языке \verb|C++| и \TeX.

\subsubsection{Поиск ошибок}

В процессе разработки использовался свободный кроссплатформенный отладчик GNU
GDB, предоставляющий широкие возможности по отслеживанию выполнения программы,
поиску ошибок и изучению её работы. Также в процессе разработки использовался
статический анализатор кода CppCheck, указывающий на множество незамеченных и
неочевидных ошибок в исходном коде на основе его анализа. Для поиска ошибок
выделения памяти использовался анализатор памяти Valgrind, результатом работы
которого является список потенциальных утечек памяти, использований
освобождённой памяти и других ошибок работы с ней. Для измерения скорости работы
частей программы и нахождения узких мест использовался комплекс для измерения
производительности Valgrind, предоставляющий табоицу функций программы со
множеством статистических и временных данных.

\subsubsection{Основная библиотека}

В качестве основной библиотеки разработки приложения использовалась библиотека с
открытым исходным Qt5,
которая позволяет разрабатывать кроссплатформенные приложения с одинаковым
программным функционалом. Библиотека в том числе позволяет разрабатывать приложения
для устройств под управлением ОС Android, на которую в основном нацелена данная работа.
Библиотека может свободно использоваться в проектах и в академических и в коммерческих целях.

Кроме того библиотека позволяет в автоматическом режиме формировать установочный пакет,
который будет использоваться при установке приложения на мобильное устройство.
Все необходимые зависимости, а так же обертки для классов языка \verb|C++| в язык Java
библиотека формирует так же сама.

При реализации нейронной сети была использована библиотека FANN\cite{fann}. FANN~--
нейросетевая библиотека с открытым исходным кодом, которая реализует многослойные искуственные
нейронные сети на языке С. Библиотека является кросс-платформенной, она проста в использовании
универсальна, хорошо документирована и быстра.

\subsubsection{Тестирование}

Для написания тестов был использован Google C++ Testing Framework, который
является библиотекой для модульного тестирования на языке \verb|C++|. Google
Test построена на методологии тестирования xUnit, то есть когда отдельные части
программы (классы, функции, модули) проверяются отдельно друг от друга, в
изоляции.

\subsubsection{Сборка и запуск разработанного программного обеспечения}

Проект собирался при помощи системы сборки проектов QMake. Она была выбрана
из-за большой гибкости, простоты подключения и линковки сторонних библиотек и
удобства и относительной простоты файлов описания процесса сборки.

Для сборки разработанного программного обеспечения необходимо склонировать
репозиторий с исходным кодом или скопировать папку с ним к себе на ПК. После этого
достаточно последовательно запустить утилиты qmake и make.

\subsubsection{Документация}

С помощью утилиты Doxygen были сгенерированы диаграммы, отображающие
взаимосвязи исходного кода. Это позволяет <<увидеть>> архитектуру,
не читая все исходные файлы проекта.

\subsubsection{Сопроводительные записки}

Техническое задание, расчётно-пояснительная записка и прочее верстались
в системе вёрстки \LaTeX. Это дало возможность сконцетрироваться на
тексте вместо его оформления, получая красивую вёрстку <<из коробки>>, множество
удобных средств для оформления текста и прочее.

\subsection{База данных}

В процессе работы программа использует СУБД SQLite~-- компактная
встраиваемая реляционная база данных с открытым кодом. SQLite является системой
управления базами данных по умолчанию в мобильных устройствах под управлением ОС Android.

Система является очень компактной, легковестной и кроссплатформенной.
Система имеет привязку к самым различным
языкам программирования, таким, как Delphi, \verb|C++|, Java, \verb|C#|, VB.NET, Python, Perl, PHP, PureBasic и другим.

СУДБ кроме того имеет набор ограничений, которые необходимо иметь в виду при разработке программного
обеспечения. Однако, на разработку данного проекта эти ограничения существенного влияния не оказали.

\subsubsection{Схема базы данных}

Схема базы данных может быть представлена ER-диаграммой, изображенной на рисунке \ref{fig:er_diagram}.

\begin{figure}[hbtp]
    \centering
    \includegraphics[width=\textwidth]{er_diagram.png}
    \caption{Схема базы данных}
    \label{fig:er_diagram}
\end{figure}

\subsubsection{Каскад Хаара}

В процессе работы приложения, алгоритм Виолы-Джонса использует стандартный каскад Хаара \cite{haar_cascade},
который был обучен ранее на большом объеме тестовых изображений.
Стандартный каскад представлен XML-документом и свободно доступен для загрузки.
Для более производительной работы приложения, каскад был перенесен из формата
XML в базу данных с помощью Perl-скрипта.

Приведем часть первого \textit{stage} из стандартного каскада:
\begin{verbatim}
<haarcascade_frontalface_alt>
  <size>20 20</size>
  <stages>
    <_>
      <!-- stage 0 -->
      <trees>
        <_>
          <!-- tree 0 -->
          <_> 
            <!-- root node -->
            <feature>
              <rects>
                <_>3 7 14 4 -1.</_>
                <_>3 9 14 2 2.</_></rects>
              <tilted>0</tilted></feature>
            <threshold>4.0141958743333817e-003</threshold>
            <left_val>0.0337941907346249</left_val>
            <right_val>0.8378106951713562</right_val></_></_>
        <_>
          <!-- tree 1 -->
          <_>
            <!-- root node -->
            <feature>
              <rects>
                <_>1 2 18 4 -1.</_>
                <_>7 2 6 4 3.</_></rects>
              <tilted>0</tilted></feature>
            <threshold>0.0151513395830989</threshold>
            <left_val>0.1514132022857666</left_val>
            <right_val>0.7488812208175659</right_val></_></_>
        ...
      # после всех признаков (деревьев) ...
      <stage_threshold>6.9566087722778320</stage_threshold>
      <parent>0</parent>
      <next>-1</next></_>
      ...
\end{verbatim}

Для хранения каскада Хаара в приложении были использованы следующие таблицы из
схемы, изображенной на рисунке \ref{fig:er_diagram}:

\begin{description}
    \item[rects] \hfill \\
        Таблица содержит массив прямоугольников, которые характеризуют
        конкретный признак. Так как каждый примитив Хаара представляет собой комбинацию из
        областей черного и белого цвета (рис. \ref{fig:haar}), при чем область черного цвета только одна,
        то для кодирования одного примитива достаточно следующих полей:
        \begin{description}
            \item[x] -- определяет горизонтальную координату темной части признака;
            \item[y] -- определяет вертикальную координату темной части признака;
            \item[width] -- определяет ширину темной части признака;
            \item[height] -- определяет высоту темной части признака.
        \end{description}
        Кроме того, при расчетах, используемых в алгоритме Виолы-Джонса используется
        вес каждого прямоугольного признака, который определен в поле \textbf{weight} таблицы.
        Каждый признак в каскаде Хаара может иметь более чем один примитив. Поэтому, в поле \textbf{feature\_id}
        для каждого примитива определено, к какому признаку он относится;
    \item[nodes] \hfill \\
        Таблица содержит список узлов каскада Хаара. Каждый узел характеризуется набором характеристик, среди
        которых выделяются:
        \begin{description}
            \item[threshold]~-- предельное значение, при преодолении которого узел можно считать отвергнутым;
            \item[left\_val]~-- значение, на которое будет увеличена характеристика признака в случае, если узел был отвергнут;
            \item[right\_val]~-- значение, на которое будет увеличена характеристика признака в случае, если узел был принят; 
            \item[feature\_id]~-- индекс признака, к которому относится данный узел.
        \end{description}
    \item[features] \hfill \\
        Таблица содержит список признаков, которыми оперирует алгоритм Виолы-Джонса.
        Каждый признак в общем случае может содержать только один узел, но при этом он содержит в себе
        множество примитивов. Каждый признак является частью одного этапа распознавания. При этом признаков в одном этапе
        может быть много. Идентификатор этапа хранится в поле \textbf{stage\_id}. Каждый признак кроме того может быть под
        наклоном. Этот факт кодируется в поле \textbf{tilted}. Так как в данном приложении используются только основные признаки
        Хаара, это поле всегда содержит 0. Однако, так как предполагается дальнейшая доработка программного продукта с
        добавлением дополнительных признаков, поле имеет место быть;
    \item[stages] \hfill \\
        Таблица содержит список этапов распознавания. Каждый этап состоит из набора признаков.
        В ходе алгоритма происходит накопление суммарных значений признаков на каждом этапе.
        В случае, если накопленное значение превысило значение \textbf{threshold}, этап отвергается и алгоритм
        переходит к следующему сканирующему окну. Для навигации по каскаду, используются поля
        \textbf{parent\_stage} и \textbf{next\_stage}. Кроме того, каждый этап ссылается на таблицу \textbf{sizes},
        определяя тем самым масштаб, который необходимо применить к сканируемому окну;
    \item[sizes] \hfill \\
        Таблица содержит список размеров примитивов. При этом в поле \textbf{width} определена ширина примитива, а в
        поле \textbf{height}~-- высота. Потенциально, алгоритм Виолы-Джонса способен работать с примитивами различного размера,
        повышая тем самым детализацию признаков. Однако, в текущей реализации программы используется лишь примитивы размера
        $20\times20$ точек и таблица была создана с целью увеличения функционала в будущем.
\end{description}

\subsubsection{Анализируемые объекты}

В ходе своей работы, программный продукт использует нейронную сеть для определения принадлежности анализируемого
изображения конкретной личности. Сами же личности должны быть сохранены в базе данных для последующего использования.

Именно эти данные хранит таблица \textbf{names}, при чем в поле \textbf{name} хранится полная информация о человеке.
Так как приложение не подразумевает, что для идентификации человека будет использовано именно его имя, то одного поля вполне
достаточно.

\subsubsection{Тестовые данные}

В ходе тестирования приложения был использован набор изображений из базы изображений под названием
\textit{Yale Face Database B} \cite{yale_face_database}. Она содержит более 16000 изображений 28ми человек в различных позах
и с различным углом наклона головы. Для хранения информации о местонахождении изображений на жестком диске
компьютера, на котором производится тестирование, используется таблица \textbf{test\_images}.

\subsection{Модульное тестирование}
Для разработанного программного обеспечения были написаны модульные тесты,
позволяющие проверить корректность ранее оттестированного кода при внесении
изменений. Пример прогонки модульных тестов показан на рисунке
\ref{fig:test}. Пример исходного кода модульных тестов представлен в приложенни
\ref{ap1}.

\begin{figure}[hbt]
  \centering
  \includegraphics{tests.png}
  \caption{Пример прогона модульных тестов}
  \label{fig:test}
\end{figure}

\subsection{Структура программы}

Здесь мы опишем структуру программы, её основные элементы и устройство.
На рисунке \ref{fig:proj-struct} увидеть из каких модулей состоит программа.

\begin{figure}[h!]
  \centering
  \includegraphics{uml.png}
  \caption{Структура программы}
  \label{fig:proj-struct}
\end{figure}

\subsection*{Выводы}
\addcontentsline{toc}{subsection}{Выводы}

Были рассмотрены основные методы, которые легли в основу разработанного
программного продукта. Была представлена и полностью описана схема базы данных,
была представлена диаграмма, показывающая структуру приложения.

\clearpage 
\section{Экспериментальный раздел}

В данном разделе будут представлены эксперименты, которые показывают
работоспособность созданного программного обеспечения.

\subsection{Достоверность распознавания}

Целью экспериментов является определение достоверности распознавания
разработанным программным продуктом. Достоверность распознавания определяется по
формуле
\[ \frac{TP}{TP+FP}, \]
где $TP$ - число правильно определенных объектов, $FP$ - число ложных обнаружений.


В эксперименте учавствуют 400 изображений, принадлежащие 40 людям.  Одно из
десяти изображений, принадлежащих одному человеку, является тестовым и не входит
в обучающую выборку.  В ходе экспериментов были получены результаты тестирования
скорости и достоверности распознавания. Результаты представлены на рисунке
\ref{fig:acc}.

\begin{figure}[h!]
  \centering
    \captionsetup{justification=centering}
  \includegraphics[width=.8\textwidth]{acc_from_iter.png}
  \caption{Исследование достоверности распознавания в зависимости от числа
максимальных итераций обучения нейронной сети.}
  \label{fig:acc}
\end{figure}

\subsection{Скорость распознавания}

Также были проведены исследования скорости обучения нейронной сети
в зависимости от максимального количества итераций. Результаты представлены на
рисунке \ref{fig:time}.

\begin{figure}[h!]
  \centering
    \captionsetup{justification=centering}
  \includegraphics[width=.8\textwidth]{time_from_iter.png}
  \caption{Исследование времени обучения в зависимости от числа максимальных 
итераций обучения нейронной сети.}
  \label{fig:time}
\end{figure}

\subsection*{Выводы}
\addcontentsline{toc}{subsection}{Выводы}

Проведенные эксмперименты показали, что увеличение числа нейронов в сети
приводит к росту времени обучения нейронной сети, однако при большем числе нейронов в скрытом
слое, достоверность распознавания увеличивается.

\clearpage

%	\section*{Заключение} \addcontentsline{toc}{section}{Заключение}

В ходе разработки проекта был изучен набор алгоритмов для распознавания лиц в
реальном времени, изучены новые возможности языка C++ и его подмножества C++11,
изучено и применено несколько сторонних библиотек, отточены навыки
программирования и использования некоторых технических средств. Итогом работы
является целиком завершённая программа, а также некоторый исследовательский
материал на её основе. В целом, работа выполнена успешно.



%	\input{appendix}
	\newpage
    \bibliographystyle{plain}
	%\addcontentsline{toc}{section}{\bibname}
	%\nocite{*}
	%\bibliography{biblio}
\end{document}
