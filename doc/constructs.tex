\subsection{Конструкторский раздел}

Lalala

\subsection{Метод Виолы-Джонса}

Хотя метод был разработан и представен еще в 2001 году Полом Виолой и Майклом
Джонсом \cite{viola_jones}, он до сих пор является основополагающим для поиска
объектов на изображении в реальном времени.
Алгоритм может распознавать различные классы изображений, однако основной задачей
при его создании было обнаружение лиц.

Существует множество реализаций этого метода, в том числе в составе библиотеки
компьютерного зрения OpenCV \cite{opencv}.

Основные принципы, на которых реализован алгоритм, таковы:
\begin{itemize}
    \item используются изображения в интегральном представлении,
        что позволяет вычислять быстро необходимые объекты;
    \item используются признаки Хаара, с помощью которых происходит
        поиск нужного объекта (в данном контексте, лица и его черт);
    \item используется бустинг (от англ. \textit{boost}~-- улучшение,
        усиление) для выбора наиболее подходящих признаков для искомого
        объекта на данной части изображения;
    \item все признаки поступают на вход классификатора,
        который даёт результат <<верно>> либо <<ложь>>;
    \item используются каскады признаков для быстрого отбрасывания окон,
        где не найдено лицо.
\end{itemize}

Обучение классификаторов идет очень медленно, но результаты поиска
лица очень быстры, именно поэтому был выбран данный метод
распознавания лиц на изображении. Виола-Джонс является
одним из лучших по соотношению показателей эффективность
распознавания/скорость работы. Также этот детектор
обладает крайне низкой вероятностью ложного обнаружения лица. Алгоритм
даже хорошо работает и распознает черты
лица под небольшим углом, примерно до 30 градусов. При угле
наклона больше 30 градусов
процент обнаружений резко падает.

Данный метод в общем виде ищет лица и черты лица по общему
принципу сканирующего окна.

\clearpage
